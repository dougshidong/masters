\documentclass[letterpaper,12pt,]{article}

\usepackage{titling}

\setlength{\droptitle}{5in}   % This is your set screw

\usepackage[%
    left=1in,%
    right=1in,%
    top=1in,%
    bottom=1.0in,%
    paperheight=11in,%
    paperwidth=8.5in%
]{geometry}%
\usepackage{comment}

\usepackage{listings}
\usepackage{amsmath}
\usepackage[section]{placeins}
\usepackage[font=small,skip=-2pt]{caption}
\usepackage{subcaption}
\usepackage{hyperref}
\usepackage{booktabs}
\usepackage{pdfpages}


\pagestyle{empty} % Remove page numbering
\linespread{1} % Line Spacing

\begin{document}

% First Order Partial Derivative
\newcommand\del[2]
{%
\dfrac{\partial#1}{\partial#2}
}

% First Order Total Derivative
\newcommand\Dt[2]
{%
\dfrac{d#1}{d#2}
}

% Math Bold Font
\newcommand\mbf[1]
{%
\mathbf{#1}
}

% Math Bold Symbol
\newcommand\mbs[1]
{%
\boldsymbol{#1}
}

% Bold x_volume
\newcommand\xvol[0]
{%
\mbf{x_{vol}}
}

% Bold x_surf
\newcommand\xsurf[0]
{%
\mbf{x_{surf}}
}

% Bold x_rbf
\newcommand\xrbf[0]
{%
\mbf{x_{rbf}}
}

% Bold alpha
\newcommand\balfa[0]
{%
\mbs{\alpha}
}

% Bold R
\newcommand\bR[0]
{%
\mbf{R}
}

% Bold w
\newcommand\bw[0]
{%
\mbf{w}
}

% Invert a Matrix with Square Brackets
\newcommand\inv[1]
{%
\left[#1\right]^{-1}
}

%Square Brackets
\newcommand\sqbr[1]
{%
\left[#1\right]
}

% N_xvol
\newcommand\nrbf[0]
{%
N_{rbf}
}

% N_xvol
\newcommand\nxvol[0]
{%
N_{xvol}
}

% N_cell
\newcommand\ncell[0]
{%
N_{cell}
}

% N_des
\newcommand\ndes[0]
{%
N_{des}
}

% Bold psi
\newcommand\bpsi[0]
{%
\mbs{\psi}
}

% Bold eta
\newcommand\blam[0]
{%
\mbs{\lambda}
}

% Smaller font in equation
\newcommand\smt[1]
{%
\text{\footnotesize$#1$}
}
% Used to create an alignment character & under equations
\newcommand\sizealign[2]
{%
	&& \smt{\left[ #1 \times #2 \right]}
}

\section*{Mesh Movement}

The volume mesh points are moved by the RBF points which are determined by the surface points.
The surface points are parametrized through the design variables.
Therefore, the volume mesh points are a function of the design variables.

\begin{equation}
	\xvol 
	= 
	\xvol(
	\xrbf(
	\xsurf(
	\balfa))) 
	= 
	\xvol(\balfa)
\end{equation}

The RBF method has the form in Equation \ref{eq:rbf}, which can be written in a system of linear equation.
The weights can be solved by using the displacement of the RBF points as shown in Equation \ref{eq:solveRBF}.
The displacements of the volume mesh points are found using weights as shown in Equation \ref{eq:disprbf}.

\begin{equation}
	f(x) = \sum_{i=1}^n w_i \phi(\|x - x_i\|)
	\label{eq:rbf}
\end{equation}

\begin{equation}
	\Delta \xrbf
		= \mbf{M} 
		  \bw
	\label{eq:solveRBF}
\end{equation}


\begin{equation}
	\Delta \xrbf 
		= \mbf{A} 
		  \bw
		= \mbf{A}
		  \inv{\mbf{M}}
		  \Delta \xrbf
	\label{eq:disprbf}
\end{equation}

Since every processor has all the necessary information to move its own volume points, the mesh deformation can be done fully in parallel.
Finally, it is possible to define the mesh sensitivities with Equation \ref{eq:meshsens} \& \ref{eq:fullmeshsens}.

\begin{equation}
	\Dt{\xvol}{\xrbf}
	= \dfrac{\Delta \xvol}{\Delta \xrbf}
		= \mbf{A}
		  \inv{\mbf{M}}
	\label{eq:meshsens}
\end{equation}

\begin{equation}
	\Dt{\xvol}{\balfa}
	=
	\Dt{\xvol}{\xrbf}
	\Dt{\xrbf}{\xsurf}
	\Dt{\xsurf}{\balfa}
	=
	\mbf{A}
	\inv{\mbf{M}}
	\Dt{\xrbf}{\xsurf}
	\Dt{\xsurf}{\balfa}
	\label{eq:fullmeshsens}
\end{equation}

\section*{First-Order Sensitivities}

\subsection*{Flow Sensitivities}

The cost function $L$ in Equation \ref{eq:cost} is a function of the flow state variables $\bw$ and the geometry $\xvol$.
The state variables are constrained by the steady-state solution of the Navier-Stokes, where the residual is zero as shown in Equation \ref{eq:flowres}.

\begin{equation}
	L = L(\bw, \xvol)
	\label{eq:cost}
\end{equation}

\begin{equation}
	\bR = \bR(\bw, \xvol) = \mbf{0}
	\label{eq:flowres}
\end{equation}

The gradient of the cost function and the flow residual are defined through the chain rule in Equation \ref{eq:dIdDes_plain} \& \ref{eq:dRdDes_plain}.

\begin{equation}
	\Dt{L( \bw, \xvol ) } {\balfa}
	= 
    \del{L}{\bw}
	\Dt{\bw}{\balfa}
	+
	\del{L}{\mbf{x}_{vol}}
	\Dt{\mbf{x}_{vol}}{\balfa}
	\label{eq:dIdDes_plain}
\end{equation}

\begin{equation}
	\Dt{\bR( \bw, \xvol ) } {\balfa}
	= 
    \del{\bR}{\bw}
	\Dt{\bw}{\balfa}
	+
	\del{\bR}{\mbf{x}_{vol}}
	\Dt{\mbf{x}_{vol}}{\balfa}
	=
	\mbf{0}
	\label{eq:dRdDes_plain}
\end{equation}

\subsubsection*{Direct Differentiation}

Equation \ref{eq:dRdDes_plain} is used to compute the derivative of the state variables with respect to the design variables ${d\bw}/{d\balfa}$.
The Jacobian of the residual is required and the linear system is solved for $\ndes$ RHS vectors.

\begin{equation}
\begin{alignedat}{5}
	&\del{\bR}{\bw}
	&&\Dt{\bw}{\balfa}
	&&=
    &&-
	\del{\bR}{\mbf{x}_{vol}}
	&&\Dt{\mbf{x}_{vol}}{\balfa}
	\\
	& \smt{\left[ \ncell \times \ncell \right]}
	\sizealign{\ncell}{\ndes}
	&&=
	\sizealign{\ncell}{\nxvol}
	\sizealign{\nxvol}{\ndes}
	\label{eq:dWdDes}
\end{alignedat}
\end{equation}


\subsubsection*{Adjoint Variable}

It is possible to augment the gradient of the cost function with an adjoint variable that multiplies the gradient of residual since it is zero.
Equation \ref{eq:adjointDerivation} shows the augmented cost function being re-arranged into flow and metric contributions.


\begin{equation}
\begin{split}
	\Dt{L_a( \bw, \xvol )}{\balfa} &= 
	\Dt{L( \bw, \xvol )}{\balfa} +
	\bpsi^T
	\Dt{\bR( \bw, \xvol }{\balfa})
\\
	&= 
    \del{L}{\bw}
	\Dt{\bw}{\balfa}
	+
	\del{L}{\mbf{x}_{vol}}
	\Dt{\mbf{x}_{vol}}{\balfa}
	+ 
	\bpsi^T
	\sqbr{
		\del{\bR}{\bw}
		\Dt{\bw}{\balfa}
		+
		\del{\bR}{\mbf{x}_{vol}
	}
	\Dt{\mbf{x}_{vol}}{\balfa}
	}
\\
	&=
    \underbrace
    {
		\sqbr{
			\del{L}{\bw}
			+
			\mbs{\psi}^T
			\del{\bR}{\bw}
		}
		\Dt{\bw}{\balfa}
    }
    _
    {
        \text{Flow Contributions}
    }
	+
    \underbrace
    {
		\sqbr{
			\del{L}{\mbf{x}_{vol}}
			+
			\mbs{\psi}^T
			\del{\bR}{\mbf{x}_{vol}}
		}
		\Dt{\mbf{x}_{vol}}{\balfa}
    }
    _
    {
        \text{Metric Contributions}
    }
	\label{eq:adjointDerivation}
\end{split}
\end{equation}

The goal of the adjoint variable is to avoid computing the expensive sensitivities by transfering their weights into cheaper sensitivities.
In the case of the case above, the mesh sensitivies ${d\xvol}/{d\balfa}$ have simple analytical formulas, whereas ${d\bw}/{d\balfa}$ requires solving Equation \ref{eq:dWdDes}
Therefore, the adjoint defined such that the flow contribution is zero by solving Equation \ref{eq:adjoint}.

\begin{equation}
	\bR^{\bpsi} =
	\del{L}{\bw}
	+
	\mbs{\psi}^T
	\del{\bR}{\bw}
	= \mbf{0}
\end{equation}
\begin{equation}
\begin{alignedat}{4}
	&\sqbr{\del{\bR}{\bw}}^T
	&&\mbs{\psi}
	&&=
	&&\sqbr{\del{L}{\bw}}^T
\\	
	& \smt{\left[ \ncell \times \ncell \right]}
	\sizealign{\ncell}{1}
	&&=
	\sizealign{\ncell}{1}
\label{eq:adjoint}
\end{alignedat}
\end{equation}

Note that solving Equation \ref{eq:adjoint} requires the solution of a linear system with only 1 RHS vector as opposed to $\ndes$ for the direct differentiation method.
After solving for the adjoint, the gradient of the cost function becomes Equation \ref{eq:costAfterAdjoint}.

\begin{equation}
	\Dt{L_a}{\balfa} = 
	\sqbr{
		\del{L}{\mbf{x}_{vol}}
		+
		\mbs{\psi}^T
		\del{\bR}{\mbf{x}_{vol}}
	}
	\Dt{\mbf{x}_{vol}}{\balfa}
	\label{eq:costAfterAdjoint}
\end{equation}

\subsection*{Mesh Sensitivities}

The computation of the mesh sensitivities can be done once before the start of the design cycles since the mesh is always deformed from its initial shape.
The matrix $\mbf{A}$ and $\mbf{M}$ are constant throughout the design process.
However, as shown in Equation \ref{eq:meshsenssize}, the memory required to hold those matrices can be excessive.
For this reason, $\mbf{A}$ is never stored and lazy evaluation is required.

Unfortunately, the factorization of $\mbf{M}$ is required by every processor.
Typical number of RBF points used is in the range of $2,000-10,000$.
Note that $12,000$ RBF points will require 1.1 Gb of storage in each processor.
Also, at every design cycle, the multiplication of $\mbf{A}$ has to be done with a matrix with $\ndes$ columns.

\begin{equation}
\begin{alignedat}{6}
	&\Dt{\xvol}{\balfa}
	&&=
	&& \mbf{A}
	&& \inv{\mbf{M}}      
	&& \Dt{\xrbf}{\xsurf} 
	&& \Dt{\xsurf}{\balfa}
\\	                   
	&\smt{\left[\nxvol \ndes \right]}
	&&=
	\sizealign{\nxvol}{\nrbf}
	\sizealign{\nrbf}{\nrbf}
	\sizealign{\nrbf}{N_{surf}}
	\sizealign{N_{surf}}{\ndes}
\label{eq:meshsenssize}
\end{alignedat}
\end{equation}

One way to reduce CPU time of the computation of the gradient is to introduce another adjoint variable.
This time, the cost function is augmented in Equation \ref{eq:meshadjointDerivation} by the mesh deformation residual in \ref{eq:meshres}, which come directly from Equation \ref{eq:meshsenssize}.

\begin{equation}
	\bR^{\mbf{x}} 
	= \bR^{\mbf{x}}(\xvol) 
	= \Dt{\xvol}{\balfa} 
	- \mbf{A}
	\inv{\mbf{M}}
	\Dt{\xrbf}{\xsurf}
	\Dt{\xsurf}{\balfa}
	= \mbf{0}
	\label{eq:meshres}
\end{equation}

\begin{equation}
\begin{split}
	\Dt{L_a( \bw, \xvol )}{\balfa} &= 
	\Dt{L( \bw, \xvol )}{\balfa} 
	+
	\bpsi^T
	\Dt{\bR( \bw, \xvol }{\balfa})
	+
	\blam^T
	\Dt{\bR^{\mbf{x}}(\xvol }{\balfa})
\\
	&=
	\sqbr{
		\del{L}{\bw}
		+
		\mbs{\psi}^T
		\del{\bR}{\bw}
	}
	\Dt{\bw}{\balfa}
	\\
	&\quad+
	\sqbr{
		\del{L}{\mbf{x}_{vol}}
		+
		\mbs{\psi}^T
		\del{\bR}{\mbf{x}_{vol}}
		+
		\blam^T
	}
	\Dt{\mbf{x}_{vol}}{\balfa}
	\\
	&\quad-
	\blam^T
	\sqbr{
		\mbf{A}
		\inv{\mbf{M}}
		\Dt{\xrbf}{\xsurf}
		\Dt{\xsurf}{\balfa}
	}
	\label{eq:meshadjointDerivation}
\end{split}
\end{equation}

First, the flow adjoint is solved as Equation \ref{eq:adjoint}.
Second, the mesh adjoint can easily be solved in Equation \ref{eq:meshadjoint} as the sum of two vectors to avoid $d\xsurf/d\balfa$.

\begin{equation}
\begin{alignedat}{6}
	&\blam^T 
	&&= 
	-(&&\del{L}{\mbf{x}_{vol}}
	&&+
	&&\mbs{\psi}^T
	&&\del{\bR}{\mbf{x}_{vol}})
\\	                   
	&\smt{\left[1 \times \nxvol \right]}
	&&=
	\sizealign{1}{\nxvol}
	&&+
	\sizealign{1}{\ncell}
	\sizealign{\ncell}{\nxvol}
\label{eq:meshadjoint}
\end{alignedat}
\end{equation}

After both adjoints are solved, the gradient of the functional can be evaluated as Equation \ref{eq:gradAfter2adjoints}.
Note that it is possible to take transpose of the gradient in order to solve a smaller system of equation.

\begin{equation}
	\Dt{L_a}{\balfa}
	=
	-\blam^T
	\mbf{A}
	\inv{\mbf{M}}
	\Dt{\xrbf}{\xsurf}
	\Dt{\xsurf}{\balfa}
\label{eq:gradAfter2adjoints}
\end{equation}

Equation \ref{eq:grad2adjointsT} shows that the $\mbf{A}$ matrix multipliesa vector with 1 column as opposed to $\ndes$ and the $\mbf{M}^T$ solves for 1 column instead of $\ndes$.
Therefore, the calculation of the gradient is now completely independent of the number of design variables.

\begin{equation}
	\sqbr{\Dt{L_a}{\balfa}}^T
	=
	-\sqbr{\Dt{\xsurf}{\balfa}}^{T}
	\sqbr{\Dt{\xrbf}{\xsurf}}^{T}
	\sqbr{\mbf{M}}^{-T}
	\sqbr{\mbf{A}}^{T}
	{\blam}
	=
	-\sqbr{\Dt{\xsurf}{\balfa}}^{T}
	\sqbr{\Dt{\xrbf}{\xsurf}}^{T}
	\mbf{b}
\label{eq:grad2adjointsT}
\end{equation}

\begin{equation}
\begin{alignedat}{5}
	&\sqbr{\mbf{M}}^{T}
	&&\mbf{b}
	&&=
	&&\sqbr{\mbf{A}}^{T}
	&&{\blam}
\\	                   
	&\smt{\left[\nrbf \times \nrbf \right]}
	\sizealign{\nrbf}{1}
	&&=
	\sizealign{\nrbf}{\nxvol}
	\sizealign{\nxvol}{1}
\label{eq:meshadjointb}
\end{alignedat}
\end{equation}



\end{document}
