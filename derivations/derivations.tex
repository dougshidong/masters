\documentclass[letterpaper,12pt,]{article}

\usepackage{titling}

\setlength{\droptitle}{5in}   % This is your set screw

\usepackage[%
    left=1in,%
    right=1in,%
    top=1in,%
    bottom=1.0in,%
    paperheight=11in,%
    paperwidth=8.5in%
]{geometry}%
\usepackage{comment}

\usepackage{listings}
\usepackage{amsmath}
\usepackage[section]{placeins}
\usepackage[font=small,skip=-2pt]{caption}
\usepackage{subcaption}
\usepackage{hyperref}
\usepackage{booktabs}
\usepackage{pdfpages}


\pagestyle{empty} % Remove page numbering
\linespread{1} % Line Spacing

\begin{document}

% First Order Partial Derivative
\newcommand\del[2]{
\dfrac{\partial#1}{\partial#2}}
% Second Order Partial Derivative
\newcommand\ddel[2]{
\dfrac{\partial^2#1}{\partial#2^2}}
% Second Order Partial Derivative
\newcommand\ddelm[3]{
\dfrac{\partial^2#1}{\partial#2 \partial#3}}
% First Order Total Derivative
\newcommand\Dt[2]{
\dfrac{d#1}{d#2}}
% First Order Total Derivative
\newcommand\DDt[2]{
\dfrac{d^2#1}{d#2^2}}
% Math Bold Font
\newcommand\mbf[1]{
\mathbf{#1}}
% Math Bold Symbol
\newcommand\mbs[1]{
\boldsymbol{#1}}
% Bold x_volume
\newcommand\xvol[0]{
\mbf{x_{vol}}}
% Bold x_surf
\newcommand\xsurf[0]{
\mbf{x_{surf}}}
% Bold x_rbf
\newcommand\xrbf[0]{
\mbf{x_{rbf}}}
% Bold alpha
\newcommand\balfa[0]{
\mbs{\alpha}}
% Bold R
\newcommand\bR[0]{
\mbf{R}}
% Bold w
\newcommand\bw[0]{
\mbf{w}}
% Invert a Matrix with Square Brackets
\newcommand\inv[1]{
\left[#1\right]^{-1}}
% Transpose a Matrix with Square Brackets
\newcommand\trans[1]{
\left[#1\right]^{T}}
%Square Brackets
\newcommand\sqbr[1]{
\left[#1\right]}
% N_xvol
\newcommand\nrbf[0]{
N_{rbf}}
% N_xvol
\newcommand\nxvol[0]{
N_{xvol}}
% N_cell
\newcommand\ncell[0]{
N_{cell}}
% N_des
\newcommand\ndes[0]{
N_{des}}
% Bold psi
\newcommand\bpsi[0]{
\mbs{\psi}}
% Bold eta
\newcommand\blam[0]{
\mbs{\lambda}}
\newcommand\bgam[0]{
\mbs{\gamma}}
\newcommand\bbeta[0]{
\mbs{\beta}}
% Smaller font in equation
\newcommand\smt[1]{
\text{\footnotesize$#1$}}
% Used to create an alignment character & under equations
\newcommand\nxn[2]{
	\smt{\left[ #1 \times #2 \right]}}
\newcommand\nxnxn[3]{
	\smt{\left[ #1 \times #2 \times #3\right]}}
\newcommand\blue[1]{
	{\color{blue}#1}}
\newcommand\red[1]{
	{\color{red}#1}}
\newcommand\green[1]{
	{\color{green}#1}}

\section*{Mesh Movement}

The volume mesh points are moved by the RBF points which are determined by the surface points.
The surface points are parametrized through the design variables.
Therefore, the volume mesh points are a function of the design variables.

\begin{equation}
	\xvol 
	= 
	\xvol(
	\xrbf(
	\xsurf(
	\balfa))) 
	= 
	\xvol(\balfa)
\end{equation}

The RBF method has the form in Equation \ref{eq:rbf}, which can be written in a system of linear equation.
The weights can be solved by using the displacement of the RBF points as shown in Equation \ref{eq:solveRBF}.
The displacements of the volume mesh points are found using weights as shown in Equation \ref{eq:disprbf}.

\begin{equation}
	f(x) = \sum_{i=1}^n w_i \phi(\|x - x_i\|)
	\label{eq:rbf}
\end{equation}

\begin{equation}
	\Delta \xrbf
		= \mbf{M} 
		  \bw
	\label{eq:solveRBF}
\end{equation}


\begin{equation}
	\Delta \xrbf 
		= \mbf{A} 
		  \bw
		= \mbf{A}
		  \inv{\mbf{M}}
		  \Delta \xrbf
	\label{eq:disprbf}
\end{equation}

Since every processor has all the necessary information to move its own volume points, the mesh deformation can be done fully in parallel.
Finally, it is possible to define the mesh sensitivities with Equation \ref{eq:meshsens} \& \ref{eq:fullmeshsens}.

\begin{equation}
	\Dt{\xvol}{\xrbf}
	= \dfrac{\Delta \xvol}{\Delta \xrbf}
		= \mbf{A}
		  \inv{\mbf{M}}
	\label{eq:meshsens}
\end{equation}

\begin{equation}
	\Dt{\xvol}{\balfa}
	=
	\Dt{\xvol}{\xrbf}
	\Dt{\xrbf}{\xsurf}
	\Dt{\xsurf}{\balfa}
	=
	\mbf{A}
	\inv{\mbf{M}}
	\Dt{\xrbf}{\xsurf}
	\Dt{\xsurf}{\balfa}
	\label{eq:fullmeshsens}
\end{equation}

\section*{First-Order Sensitivities}

\subsection*{Flow Sensitivities}

The cost function $L$ in Equation \ref{eq:cost} is a function of the flow state variables $\bw$ and the geometry $\xvol$.
The state variables are constrained by the steady-state solution of the Navier-Stokes, where the residual is zero as shown in Equation \ref{eq:flowres}.

\begin{equation}
	L = L(\bw, \xvol)
	\label{eq:cost}
\end{equation}

\begin{equation}
	\bR = \bR(\bw, \xvol) = \mbf{0}
	\label{eq:flowres}
\end{equation}

The gradient of the cost function and the flow residual are defined through the chain rule in Equation \ref{eq:dIdDes_plain} \& \ref{eq:dRdDes}.

\begin{equation}
	\Dt{L( \bw, \xvol ) } {\balfa}
	= 
    \del{L}{\bw}
	\Dt{\bw}{\balfa}
	+
	\del{L}{\xvol}
	\Dt{\xvol}{\balfa}
	\label{eq:dIdDes_plain}
\end{equation}

\begin{equation}
	\Dt{\bR( \bw, \xvol ) } {\balfa}
	= 
    \del{\bR}{\bw}
	\Dt{\bw}{\balfa}
	+
	\del{\bR}{\xvol}
	\Dt{\xvol}{\balfa}
	=
	\mbf{0}
	\label{eq:dRdDes}
\end{equation}

\subsubsection*{Direct Differentiation}

Equation \ref{eq:dRdDes} is used to compute the derivative of the state variables with respect to the design variables ${d\bw}/{d\balfa}$.
The Jacobian of the residual is required and the linear system is solved for $\ndes$ RHS vectors.

\begin{equation}
\begin{alignedat}{5}
	&\del{\bR}{\bw}
	&&\Dt{\bw}{\balfa}
	&&=
    &&-
	\del{\bR}{\xvol}
	&&\Dt{\xvol}{\balfa}
	\\
	& \smt{\left[ \ncell \times \ncell \right]}
	&&\nxn{\ncell}{\ndes}
	&&=
	&&\nxn{\ncell}{\nxvol}
	&&\nxn{\nxvol}{\ndes}
	\label{eq:dWdDes}
\end{alignedat}
\end{equation}


\subsubsection*{Adjoint Variable}

It is possible to augment the gradient of the cost function with an adjoint variable that multiplies the gradient of residual since it is zero.
Equation \ref{eq:adjointDerivation} shows the augmented cost function being re-arranged into flow and metric contributions.


\begin{equation}
\begin{split}
	\Dt{L_a( \bw, \xvol )}{\balfa} &= 
	\Dt{L( \bw, \xvol )}{\balfa} +
	\bpsi^T
	\Dt{\bR( \bw, \xvol }{\balfa})
\\
	&= 
    \del{L}{\bw}
	\Dt{\bw}{\balfa}
	+
	\del{L}{\xvol}
	\Dt{\xvol}{\balfa}
	+ 
	\bpsi^T
	\sqbr{
		\del{\bR}{\bw}
		\Dt{\bw}{\balfa}
		+
		\del{\bR}{\xvol
	}
	\Dt{\xvol}{\balfa}
	}
\\
	&=
    \underbrace
    {
		\sqbr{
			\del{L}{\bw}
			+
			\bpsi^T
			\del{\bR}{\bw}
		}
		\Dt{\bw}{\balfa}
    }
    _
    {
        \text{Flow Contributions}
    }
	+
    \underbrace
    {
		\sqbr{
			\del{L}{\xvol}
			+
			\bpsi^T
			\del{\bR}{\xvol}
		}
		\Dt{\xvol}{\balfa}
    }
    _
    {
        \text{Metric Contributions}
    }
	\label{eq:adjointDerivation}
\end{split}
\end{equation}

The goal of the adjoint variable is to avoid computing the expensive sensitivities by transferring their weights into cheaper sensitivities.
In the case of the case above, the mesh sensitivities ${d\xvol}/{d\balfa}$ have simple analytical formulas, whereas ${d\bw}/{d\balfa}$ requires solving Equation \ref{eq:dWdDes}
Therefore, the adjoint defined such that the flow contribution is zero by solving Equation \ref{eq:adjoint}.

\begin{equation}
\begin{split}
	\bR^{\bpsi} =
	\del{L}{\bw}
	+
	\bpsi^T
	\del{\bR}{\bw}
	= \mbf{0}
	\\
\begin{alignedat}{4}
	&\sqbr{\del{\bR}{\bw}}^T
	&&\bpsi
	&&=
	&&\sqbr{\del{L}{\bw}}^T
\\	
	& \smt{\left[ \ncell \times \ncell \right]}
	&&\nxn{\ncell}{1}
	&&=
	&&\nxn{\ncell}{1}
\label{eq:adjoint}
\end{alignedat}
\end{split}
\end{equation}

Note that solving Equation \ref{eq:adjoint} requires the solution of a linear system with only 1 RHS vector as opposed to $\ndes$ for the direct differentiation method.
After solving for the adjoint, the gradient of the cost function becomes Equation \ref{eq:dIdDesAfterAdjoint}.

\begin{equation}
	\Dt{L_a}{\balfa} = 
	\sqbr{
		\del{L}{\xvol}
		+
		\bpsi^T
		\del{\bR}{\xvol}
	}
	\Dt{\xvol}{\balfa}
	\label{eq:dIdDesAfterAdjoint}
\end{equation}

\subsection*{Mesh Sensitivities}

The computation of the mesh sensitivities can be done once before the start of the design cycles since the mesh is always deformed from its initial shape.
The matrix $\mbf{A}$ and $\mbf{M}$ are constant throughout the design process.
However, as shown in Equation \ref{eq:meshsenssize}, the memory required to hold those matrices can be excessive.
For this reason, $\mbf{A}$ is never stored and lazy evaluation is required.

Unfortunately, the factorization of $\mbf{M}$ is required by every processor.
Typical number of RBF points used is in the range of $2,000-10,000$.
Note that $12,000$ RBF points will require 1.1 Gb of storage in each processor.
Also, at every design cycle, the multiplication of $\mbf{A}$ has to be done with a matrix with $\ndes$ columns.

\begin{equation}
\begin{alignedat}{6}
	&\Dt{\xvol}{\balfa}
	&&=
	&& \mbf{A}
	&& \inv{\mbf{M}}      
	&& \Dt{\xrbf}{\xsurf} 
	&& \Dt{\xsurf}{\balfa}
\\	                   
	&\smt{\left[\nxvol \ndes \right]}
	&&=
	&&\nxn{\nxvol}{\nrbf}
	&&\nxn{\nrbf}{\nrbf}
	&&\nxn{\nrbf}{N_{surf}}
	&&\nxn{N_{surf}}{\ndes}
\label{eq:meshsenssize}
\end{alignedat}
\end{equation}

One way to reduce CPU time of the computation of the gradient is to introduce another adjoint variable.
This time, the cost function is augmented in Equation \ref{eq:meshadjointDerivation} by the mesh deformation residual in \ref{eq:meshres}, which come directly from Equation \ref{eq:meshsenssize}.

\begin{equation}
	\bR^{\mbf{x}} 
	= \bR^{\mbf{x}}(\xvol) 
	= \Dt{\xvol}{\balfa} 
	- \mbf{A}
	\inv{\mbf{M}}
	\Dt{\xrbf}{\xsurf}
	\Dt{\xsurf}{\balfa}
	= \mbf{0}
	\label{eq:meshres}
\end{equation}

\begin{equation}
\begin{split}
	\Dt{L_a( \bw, \xvol )}{\balfa} &= 
	\Dt{L( \bw, \xvol )}{\balfa} 
	+
	\bpsi^T
	\Dt{\bR( \bw, \xvol }{\balfa})
	+
	\blam^T
	\Dt{\bR^{\mbf{x}}(\xvol }{\balfa})
\\
	&=
	\sqbr{
		\del{L}{\bw}
		+
		\bpsi^T
		\del{\bR}{\bw}
	}
	\Dt{\bw}{\balfa}
	\\
	&\quad+
	\sqbr{
		\del{L}{\xvol}
		+
		\bpsi^T
		\del{\bR}{\xvol}
		+
		\blam^T
	}
	\Dt{\xvol}{\balfa}
	\\
	&\quad-
	\blam^T
	\sqbr{
		\mbf{A}
		\inv{\mbf{M}}
		\Dt{\xrbf}{\xsurf}
		\Dt{\xsurf}{\balfa}
	}
	\label{eq:meshadjointDerivation}
\end{split}
\end{equation}

First, the flow adjoint is solved as Equation \ref{eq:adjoint}.
Second, the mesh adjoint can easily be solved in Equation \ref{eq:meshadjoint} as the sum of two vectors to avoid $d\xsurf/d\balfa$.

\begin{equation}
\begin{alignedat}{6}
	&\blam^T 
	&&= 
	-(&&\del{L}{\xvol}
	&&+
	&&\bpsi^T
	&&\del{\bR}{\xvol})
\\	                   
	&\smt{\left[1 \times \nxvol \right]}
	&&=
	&&\nxn{1}{\nxvol}
	&&+
	&&\nxn{1}{\ncell}
	&&\nxn{\ncell}{\nxvol}
\label{eq:meshadjoint}
\end{alignedat}
\end{equation}

After both adjoint variables are solved, the gradient of the functional can be evaluated as Equation \ref{eq:gradAfter2adjoints}.
Note that it is possible to take transpose of the gradient in order to solve a smaller system of equation.

\begin{equation}
	\Dt{L_a}{\balfa}
	=
	-\blam^T
	\mbf{A}
	\inv{\mbf{M}}
	\Dt{\xrbf}{\xsurf}
	\Dt{\xsurf}{\balfa}
\label{eq:gradAfter2adjoints}
\end{equation}

Equation \ref{eq:grad2adjointsT} shows that the $\mbf{A}$ matrix multiplies a vector with 1 column as opposed to $\ndes$ and the $\mbf{M}^T$ solves for 1 column instead of $\ndes$.
Therefore, the calculation of the gradient is now completely independent of the number of design variables.

\begin{equation}
	\sqbr{\Dt{L_a}{\balfa}}^T
	=
	-\sqbr{\Dt{\xsurf}{\balfa}}^{T}
	\sqbr{\Dt{\xrbf}{\xsurf}}^{T}
	\sqbr{\mbf{M}}^{-T}
	\sqbr{\mbf{A}}^{T}
	{\blam}
	=
	-\sqbr{\Dt{\xsurf}{\balfa}}^{T}
	\sqbr{\Dt{\xrbf}{\xsurf}}^{T}
	\mbf{b}
\label{eq:grad2adjointsT}
\end{equation}

\begin{equation}
\begin{alignedat}{5}
	&\sqbr{\mbf{M}}^{T}
	&&\mbf{b}
	&&=
	&&\sqbr{\mbf{A}}^{T}
	&&{\blam}
\\	                   
	&\smt{\left[\nrbf \times \nrbf \right]}
	&&\nxn{\nrbf}{1}
	&&=
	&&\nxn{\nrbf}{\nxvol}
	&&\nxn{\nxvol}{1}
\label{eq:meshadjointb}
\end{alignedat}
\end{equation}

\section*{Second-Order Sensitivities}

The analytical evaluation of the Hessian has been derived by Papadimitriou and Giannokoglou.
The four different ways uses a combination of direct method and adjoint variable for the evaluation of the first and second derivatives.
\begin{itemize}
\item Direct-Direct (DD): $\ndes + \ndes(\ndes + 1)/2$
\item Adjoint-Direct (AD): $2\ndes + 1$
\item Adjoint-Adjoint (AA):$2\ndes + 1$
\item Direct-Adjoint (DA):$\ndes$
\end{itemize}

\newpage
\subsection*{Direct-Direct}

The direct-direct method uses the direct differentiation method for the first derivative as shown in Equation \ref{eq:dIdDes_plain2} \& \ref{eq:dRdDes2}.

\begin{equation}
	\Dt{L( \bw, \xvol ) } {\balfa}
	= 
    \del{L}{\bw}
	\Dt{\bw}{\balfa}
	+
	\del{L}{\xvol}
	\Dt{\xvol}{\balfa}
	\label{eq:dIdDes_plain2}
\end{equation}

\begin{equation}
	\Dt{\bR( \bw, \xvol ) } {\balfa}
	= 
    \del{\bR}{\bw}
	\Dt{\bw}{\balfa}
	+
	\del{\bR}{\xvol}
	\Dt{\xvol}{\balfa}
	= \mbf{0}
	\label{eq:dRdDes2}
\end{equation}

The full formulation of the Hessian of the cost function is shown in Equation \ref{eq:DDHessian} with its matrix sizes shown in Equation \ref{eq:DDHessiansize}.
The term $d\bw/d\balfa$ requires the solution of Equation \ref{eq:dWdDes}, which requires $\ndes$ system evaluations.

\begin{equation}
\begin{split}
	\DDt{L}{\balfa} 
	=&
	\trans{\Dt{\bw}{\balfa}}
	\ddelm{L}{\bw}{\xvol}
	\Dt{\xvol}{\balfa}
	+
	\blue{\trans{\Dt{\bw}{\balfa}}}
	\ddel{L}{\bw}
	\blue{\Dt{\bw}{\balfa}}
	+
	\del{L}{\bw}
	\red{\DDt{\bw}{\balfa}}
	\\
	&+
	\trans{\Dt{\xvol}{\balfa}}
	\ddelm{L}{\xvol}{\bw}
	\blue{\Dt{\bw}{\balfa}}
	+
	\trans{\Dt{\xvol}{\balfa}}
	\ddel{L}{\xvol}
	\Dt{\xvol}{\balfa}
	+
	\del{L}{\xvol}
	\DDt{\xvol}{\balfa}
\label{eq:DDHessian}
\end{split}
\end{equation}

\begin{equation}
\begin{split}
	\nxn{\ndes}{\ndes}
	=
	&
	\blue{\nxn{\ndes}{\ncell}}
	\nxn{\ncell}{\nxvol}
	\blue{\nxn{\nxvol}{\ndes}}
	\\&
	+
	\nxn{\ndes}{\ncell}
	\nxn{\ncell}{\ncell}
	\nxn{\ncell}{\ndes}
	\\&
	+
	\nxn{1}{\ncell}
	\red{\nxnxn{\ncell}{\ndes}{\ndes}}
	\\&
	+
	\nxn{\ndes}{\nxvol}
	\nxn{\nxvol}{\ncell}
	\blue{\nxn{\ncell}{\ndes}}
	\\&
	+
	\nxn{\ndes}{\nxvol}
	\nxn{\nxvol}{\nxvol}
	\nxn{\nxvol}{\ndes}
	\\&
	+
	\nxn{1}{\nxvol}
	\nxnxn{\nxvol}{\ndes}{\ndes}
\label{eq:DDHessiansize}
\end{split}
\end{equation}

The term $d^2\bw/d\balfa^2$ can be found in the second-order direct-differentiation of the flow residual.
The term requires the solution of Equation \ref{eq:ddWdDesdDes}, which requires $\ndes(\ndes+1)/2$ system evaluations.
\begin{equation}
\begin{split}
	\DDt{\bR}{\balfa} 
	=&
	\blue{\trans{\Dt{\bw}{\balfa}}}
	\ddelm{\bR}{\bw}{\xvol}
	\Dt{\xvol}{\balfa}
	+
	\blue{\trans{\Dt{\bw}{\balfa}}}
	\ddel{\bR}{\bw}
	\Dt{\bw}{\balfa}
	+
	\del{\bR}{\bw}
	\DDt{\bw}{\balfa}
	\\
	&+
	\trans{\Dt{\xvol}{\balfa}}
	\ddelm{\bR}{\xvol}{\bw}
	\blue{\Dt{\bw}{\balfa}}
	+
	\trans{\Dt{\xvol}{\balfa}}
	\ddel{\bR}{\xvol}
	\Dt{\xvol}{\balfa}
	+
	\del{\bR}{\xvol}
	\red{\DDt{\xvol}{\balfa}}
	=
	\mbf{0}
\label{eq:ddRdDesdDes}
\end{split}
\end{equation}

\begin{equation}
\begin{split}
	\del{\bR}{\bw}
	\red{\DDt{\bw}{\balfa}}
	=
	-
	&
	\left(
	\blue{\trans{\Dt{\bw}{\balfa}}}
	\ddelm{\bR}{\bw}{\xvol}
	\Dt{\xvol}{\balfa}
	+
	\blue{\trans{\Dt{\bw}{\balfa}}}
	\ddel{\bR}{\bw}
	\blue{\Dt{\bw}{\balfa}}
	\right.
	\\
	&+\left.
	\trans{\Dt{\xvol}{\balfa}}
	\ddelm{\bR}{\xvol}{\bw}
	\Dt{\bw}{\balfa}
	+
	\trans{\Dt{\xvol}{\balfa}}
	\ddel{\bR}{\xvol}
	\Dt{\xvol}{\balfa}
	+
	\del{\bR}{\xvol}
	\DDt{\xvol}{\balfa}
	\right)
\label{eq:ddWdDesdDes}
\end{split}
\end{equation}

\begin{equation}
	\nxn{\ncell}{\ncell}
	\red{\nxnxn{\ncell}{\ndes}{\ndes}}
	=
	\nxnxn{\ncell}{\ndes}{\ndes}
\end{equation}

A total of $\ndes + \ndes(\ndes+1)/2$ system evaluations are required.

\newpage
\subsection*{Adjoint-Direct}
The adjoint-direct method starts with the gradient of the augmented cost function derived from the adjoint in Equation \ref{eq:dIdDesAfterAdjoint2}.
It is then differentiated directly to give Equation \ref{eq:ADHessian}.

\begin{equation}
	\Dt{L_a}{\balfa} = 
	\del{L}{\xvol}
	\Dt{\xvol}{\balfa}
	+
	\bpsi^T
	\del{\bR}{\xvol}
	\Dt{\xvol}{\balfa}
	\label{eq:dIdDesAfterAdjoint2}
\end{equation}

\begin{equation}
\begin{split}
	\DDt{L_a}{\balfa} = 
	&
	\trans{\Dt{\xvol}{\balfa}}
	\ddel{L}{\xvol}
	\Dt{\xvol}{\balfa}
	+
	\trans{\Dt{\xvol}{\balfa}}
	\ddelm{L}{\xvol}{\bw}
	\blue{\Dt{\bw}{\balfa}}
	+
	\del{L}{\xvol}
	\DDt{\xvol}{\balfa}
	\\
	&+
	\trans{
	\del{\bR}{\xvol}
	\Dt{\xvol}{\balfa}
	}
	\red{\Dt{\bpsi^T}{\balfa}}
	+
	\trans{\Dt{\xvol}{\balfa}}
	\left(
	\bpsi^T
	\ddel{\bR}{\xvol}
	\right)
	\Dt{\xvol}{\balfa}
	\\
	&+
	\trans{\Dt{\xvol}{\balfa}}
	\left(
	\bpsi^T
	\ddelm{\bR}{\xvol}{\bw}
	\right)
	\blue{\Dt{\bw}{\balfa}}
	+
	\bpsi^T
	\del{\bR}{\xvol}
	\DDt{\xvol}{\balfa}
\label{eq:ADHessian}
\end{split}
\end{equation}

The first order flow sensitivities $d\bw/d\balfa$ are evaluated from Equation \ref{eq:dWdDes} and require $\ndes$ system evaluations.
The adjoint variable is evaluated in Equation \ref{eq:adjoint} at the cost of 1 system evaluation and is the reason Equation \ref{eq:dIdDesAfterAdjoint2} holds true.

That leaves $d\bpsi^T/d\balfa$ to be evaluated.
The term can be found in the direct differentiation of the adjoint residual giving Equation \ref{eq:dadjresdDes}.
Solving Equation \ref{eq:dadjdDes} costs $\ndes$ system evaluations.

\begin{equation}
\begin{split}
	\Dt{\bR^{\bpsi}}{\balfa}
	=&
	\ddelm{L}{\bw}{\xvol}
	\Dt{\xvol}{\balfa}
	+
	\ddel{L}{\bw}
	\blue{\Dt{\bw}{\balfa}}
	+
	\trans{\del{\bR}{\bw}}
	\red{\Dt{\bpsi^T}{\balfa}}
	+
	\bpsi^T
	\ddelm{\bR}{\bw}{\xvol}
	\Dt{\xvol}{\balfa}
	+
	\bpsi^T
	\ddel{\bR}{\bw}
	\blue{\Dt{\bw}{\balfa}}
	\\
	=&
	\left[
	\ddelm{L}{\bw}{\xvol}
	+
	\bpsi^T
	\ddelm{\bR}{\bw}{\xvol}
	\right]
	\Dt{\xvol}{\balfa}
	+
	\left[
		\ddel{L}{\bw}
		+
		\bpsi^T
		\ddel{\bR}{\bw}
	\right]
	\blue{\Dt{\bw}{\balfa}}
	+
	\trans{\del{\bR}{\bw}}
	\red{\Dt{\bpsi^T}{\balfa}} = \mbf{0}
\label{eq:dadjresdDes}
\end{split}
\end{equation}

\begin{equation}
\begin{alignedat}{4}
	&\trans{\del{\bR}{\bw}}
	&&\red{\Dt{\bpsi^T}{\balfa}}
	=
	-\left(
	\left[
	\ddelm{L}{\bw}{\xvol}
	+
	\bpsi^T
	\ddelm{\bR}{\bw}{\xvol}
	\right]
	\right.&&
	\left.
	\Dt{\xvol}{\balfa}
	+
	\left[
		\ddel{L}{\bw}
		+
		\bpsi^T
		\ddel{\bR}{\bw}
	\right]
	\blue{\Dt{\bw}{\balfa}}
	\right)
\\
	& \nxn{\ncell}{\ncell}
	&& \nxn{\ncell}{\ndes}
	=
	&& \nxn{\ncell}{\ndes}
\label{eq:dadjdDes}
\end{alignedat}
\end{equation}

A total of $2\ndes+1$ system evaluations are required to compute the Hessian using the adjoint-direct approach.

\newpage
\subsection*{Adjoint-Adjoint}
Looking at the previous method, it is possible to add adjoint variables to eliminate some expensive terms by augmenting derivative of the augmented cost function like in Equation \ref{eq:ADHessian}.

\begin{equation}
	\DDt{L_a^*}{\balfa} = 
	\DDt{L_a}{\balfa}
	+
	\bbeta^T
	\Dt{\bR}{\balfa}
	+
	\bgam^T
	\Dt{\bR^{\bpsi}}{\balfa}
\label{eq:augmentedHessian_simple}
\end{equation}

The expression can be fully expanded by inserting Equation \ref{eq:ADHessian}, \ref{eq:dRdDes} and \ref{eq:dadjresdDes} into Equation \ref{eq:augmentedHessian_simple}.
The terms are then factored and gives Equation \ref{eq:augmentedHessian_exp}

\begin{equation}
\begin{split}
	\DDt{L_a^*}{\balfa} = 
	&
	\del{L}{\xvol}
	\DDt{\xvol}{\balfa}
	+
	\bpsi^T
	\del{\bR}{\xvol}
	\DDt{\xvol}{\balfa}
	\\&
	+
%%%%%%%%
	\left\{
		\trans{\Dt{\xvol}{\balfa}}
		\ddel{L}{\xvol}
		+
		\trans{\Dt{\xvol}{\balfa}}
		\left(
		\bpsi^T
		\ddel{\bR}{\xvol}
		\right)
		+
		\bbeta^T
			\del{\bR}{\xvol}
		\right.\\&\left. \quad\quad
		+
		\Dt{\xvol}{\balfa}
		\bgam^T
		\ddelm{L}{\bw}{\xvol}
		+
		\bgam^T
		\bpsi^T
		\ddelm{\bR}{\bw}{\xvol}
	\right\}
	\Dt{\xvol}{\balfa}
%%%%%%%%
	\\&+
	\left\{
		\trans{\Dt{\xvol}{\balfa}}
		\ddelm{L}{\xvol}{\bw}
		+
		\trans{\Dt{\xvol}{\balfa}}
		\left(
		\bpsi^T
		\ddelm{\bR}{\xvol}{\bw}
		\right)
	\right.\\&\left. \quad\quad
		+
		{\color{cyan}\bbeta}^T
		\del{\bR}{\bw}
		+
		\bgam^T
		\ddel{L}{\bw}
		+
		\bgam^T
		\left(
		\bpsi^T
		\ddel{\bR}{\bw}
		\right)
	\right\}
	\blue{\Dt{\bw}{\balfa}}
	\\&
	+
%%%%%%%%
	\left\{
		\trans{
		\del{\bR}{\xvol}
		\Dt{\xvol}{\balfa}
		}
		+
		\green{\bgam}^T
		\trans{\del{\bR}{\bw}}
	\right\}
	\red{\Dt{\bpsi^T}{\balfa}}
\label{eq:augmentedHessian_exp}
\end{split}
\end{equation}

The derivative of the flow adjoint can be eliminated by defining $\bgam$ to make the last term zero.
Unfortunately, as seen in Equation \ref{eq:AAsolve1}, the cost of eliminating $d\bpsi^T/d\balfa$ is the same as solving for it.
Evaluating $\bgam$ costs $\ndes$ system evaluations.

\begin{equation}
\begin{alignedat}{5}
	&\del{\bR}{\bw}
	&&\green{\bgam}
	&&=
	&&\del{\bR}{\xvol}
	&&\Dt{\xvol}{\balfa}
	\\
	& \nxn{\ncell}{\ncell}
	&&\nxn{\ncell}{\ndes}
	&&=
	&&\nxn{\ncell}{\nxvol}
	&&\nxn{\nxvol}{\ndes}
\label{eq:AAsolve1}
\end{alignedat}
\end{equation}

The same unfortunate scenario happens for $d\bw/d\balfa$ where eliminating it costs $\ndes$ flow evaluations as seen in Equation \ref{eq:AAsolve2}

\begin{equation}
\begin{alignedat}{5}
	&\del{\bR}{\bw}^T
	&&{\color{cyan}\bbeta}
	&&= -
	\left\{
	\trans{\Dt{\xvol}{\balfa}}
	\right. && \left.
	\ddelm{L}{\xvol}{\bw}
	+ ...
	\right\}^T
	\\
	& \nxn{\ncell}{\ncell}
	&&\nxn{\ncell}{\ndes}
	&&=
	&&\nxn{\ncell}{\ndes}
\label{eq:AAsolve2}
\end{alignedat}
\end{equation}

Finally, the adjoint-adjoint Hessian after solving for all the adjoint variables reduces to Equation \ref{eq:AAHessian}.

\begin{equation}
\begin{split}
	\DDt{L_a^*}{\balfa} = 
	&
	\del{L}{\xvol}
	\DDt{\xvol}{\balfa}
	+
	\bpsi^T
	\del{\bR}{\xvol}
	\DDt{\xvol}{\balfa}
	\\&
	+
%%%%%%%%
	\left\{
		\trans{\Dt{\xvol}{\balfa}}
		\ddel{L}{\xvol}
		+
		\trans{\Dt{\xvol}{\balfa}}
		\left(
		\bpsi^T
		\ddel{\bR}{\xvol}
		\right)
		+
		\bbeta^T
			\del{\bR}{\xvol}
		\right.\\&\left. \quad\quad
		+
		\Dt{\xvol}{\balfa}
		\bgam^T
		\ddelm{L}{\bw}{\xvol}
		+
		\bgam^T
		\bpsi^T \ddelm{\bR}{\bw}{\xvol}
	\right\}
	\Dt{\xvol}{\balfa}
\label{eq:AAHessian}
\end{split}
\end{equation}

A total of $2\ndes + 1$ system evaluations are required to solve every adjoint.

\newpage
\subsection*{Direct-Adjoint}

The direct-adjoint method is directly linked to the direct-direct approach.
The Hessian of the cost function expression is augmented with the Hessian of the flow residual as seen in Equation \ref{eq:DAHessian_simp}.
Equation \ref{eq:DDHessian} \& \ref{eq:ddRdDesdDes}.

\begin{equation}
\begin{split}
	\DDt{L^{**}}{\balfa} 
	=
	\DDt{L}{\balfa} 
	+
	\bpsi^T
	\DDt{\bR}{\balfa} 
\label{eq:DAHessian_simp}
\end{split}
\end{equation}

\begin{equation}
\begin{split}
	\DDt{L**}{\balfa} 
	=&
	\trans{\Dt{\bw}{\balfa}}
	\left\{
		\ddelm{L}{\bw}{\xvol}
		+
		\bpsi^T
		\ddelm{\bR}{\bw}{\xvol}
	\right\}
	\Dt{\xvol}{\balfa}
	+
	\blue{\trans{\Dt{\bw}{\balfa}}}
	\left\{
		\ddel{L}{\bw}
		+
		\bpsi^T
		\ddel{\bR}{\bw}
	\right\}
	\blue{\Dt{\bw}{\balfa}}
	\\&+
	\left\{
		\del{L}{\bw}
		+
		\green{\bpsi^T}
		\del{\bR}{\bw}
	\right\}
	\red{\DDt{\bw}{\balfa}}
	+
	\trans{\Dt{\xvol}{\balfa}}
	\left\{
		\ddelm{L}{\xvol}{\bw}
		+
		\bpsi^T
		\ddelm{\bR}{\xvol}{\bw}
	\right\}
	\blue{\Dt{\bw}{\balfa}}
	\\&+
	\trans{\Dt{\xvol}{\balfa}}
	\left\{
		\ddel{L}{\xvol}
		+
		\bpsi^T
		\ddel{\bR}{\xvol}
	\right\}
	\Dt{\xvol}{\balfa}
	+
	\left\{
		\del{L}{\xvol}
		+
		\bpsi^T
		\del{\bR}{\xvol}
	\right\}
	\DDt{\xvol}{\balfa}
\label{eq:DAHessian}
\end{split}
\end{equation}

Looking at Equation \ref{eq:DAHessian}, the expensive $d^2\bw/d\balfa^2$ term can now be ignored by solving the adjoint.
Furthermore, it only requires 1 system evaluation as seen in Equation \ref{eq:adjoint}.
The first-order flow sensitivities still have to be solved at the cost of $\ndes$ from Equation \ref{eq:dWdDes}.
Therefore, the total cost of $\ndes + \ndes(\ndes+1)/2$ system evaluations are required, making it the cheapest of the four methods.

\end{document}
